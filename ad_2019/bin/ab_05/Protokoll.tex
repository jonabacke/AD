\documentclass[12pt,a4paper]{article}
\usepackage[utf8]{inputenc}
\usepackage[german]{babel}
\usepackage[T1]{fontenc}
\usepackage{amsmath}
\usepackage{amsfonts}
\usepackage{amssymb}
\usepackage{graphicx}
\author{Fabian Erdmann, Jonathan Backes}
\title{Performanz von Sortieralgorithmen}
\begin{document}
\maketitle
\tableofcontents
\newpage
\section{Einleitung}
Sortieralgorithmen werden in der Informatik oft benutzt um zum Beispiel Daten aus einer Datenbank oder andere Tabellen zu sortieren. Hierbei ist es wichtig welche Algorithmen sich für welche Situation eignen. Dafür werden hier verschiedene Algorithmen implementiert und getestet. Die getesteten Algoritman sind der Quicksort-Algorithmus mit drei verschiedenen Pivot-Elementen, der Merge-Sort, der Insert-Sort und ein Heap-Sort-Algorithmus. Verglichen werden diese zum einen mit Countern, die die Aufrufe zählen und mit der Zeit die sie brauchen. Als Ergebnis kommt heraus, dass der Heap-Sort und der Quicksort mit dem Median Pivot die einzigen sind, die für eine sehr hohe Anzahl an Elementen keinen Overflow haben und somit die einzigen sind die dann noch in frage kommen. Auch sind diese beiden die schnellsten wobei der Heap-Sort zwar weniger Schritte braucht der Quicksort aber weniger Zeit.
\section{Theorie}
\end{document}